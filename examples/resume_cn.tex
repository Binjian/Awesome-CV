%!TEX TS-program = xelatex
%!TEX encoding = UTF-8 Unicode
% Awesome CV LaTeX Template for CV/Resume
%
% This template has been downloaded from:
% https://github.com/posquit0/Awesome-CV
%
% Author:
% Claud D. Park <posquit0.bj@gmail.com>
% http://www.posquit0.com
%
% Template license:
% CC BY-SA 4.0 (https://creativecommons.org/licenses/by-sa/4.0/)
%


%-------------------------------------------------------------------------------
% CONFIGURATIONS
%-------------------------------------------------------------------------------
% A4 paper size by default, use 'letterpaper' for US letter
\documentclass[11pt, a4paper]{awesome-cv-cn}

\usepackage{xeCJK}
\newfontfamily\hei{Adobe Heiti Std}
\newfontfamily\kai{Adobe Kaiti Std}
\setCJKmainfont[BoldFont=LXGWWenKai-Bold,ItalicFont=LXGWWenKai-Light]{LXGW WenKai}   % 设置默认中文字体
\setCJKmonofont{LXGWWenKaiMono}   % 设置等宽字体
\fontdir[fonts/]

\usepackage[style=numeric,defernumbers=true,maxbibnames=99,sorting=ydnt]{biblatex}
\usepackage{subfiles}
\usepackage{twemojis}
\usepackage{fontawesome5}
% Configure page margins with geometry
\geometry{left=1.4cm, top=.8cm, right=1.4cm, bottom=1.8cm, footskip=.5cm}


% Color for highlights
% Awesome Colors: awesome-emerald, awesome-skyblue, awesome-red, awesome-pink, awesome-orange
%                 awesome-nephritis, awesome-concrete, awesome-darknight
\colorlet{awesome}{awesome-red}
% Uncomment if you would like to specify your own color
% \definecolor{awesome}{HTML}{3E6D9C}

% Colors for text
% Uncomment if you would like to specify your own color
% \definecolor{darktext}{HTML}{414141}
% \definecolor{text}{HTML}{333333}
% \definecolor{graytext}{HTML}{5D5D5D}
% \definecolor{lighttext}{HTML}{999999}
% \definecolor{sectiondivider}{HTML}{5D5D5D}

% Set false if you don't want to highlight section with awesome color
\setbool{acvSectionColorHighlight}{true}

% If you would like to change the social information separator from a pipe (|) to something else
\renewcommand{\acvHeaderSocialSep}{\quad\textbar\quad}


%-------------------------------------------------------------------------------
%	PERSONAL INFORMATION
%	Comment any of the lines below if they are not required
%-------------------------------------------------------------------------------
% Available options: circle|rectangle,edge/noedge,left/right
\photo[rectangle,edge,right]{avatar1.jpg}
\name{忻}{斌健}
\position{深度学习{\enskip\cdotp\enskip}自动驾驶{\enskip\cdotp\enskip}软件设计}
\address{上海市青浦区巷居路99弄288号102室}

\mobile{(+86) 139-1896-1550}
\email{binjian.xin@hotmail.com}
%\dateofbirth{January 1st, 1970}
\homepage{binjian.github.io}
\github{binjian}
\linkedin{binjian-xin}
% \twitter{xinbinjian}
% \gitlab{gitlab-id}
% \stackoverflow{SO-id}{SO-name}
% \twitter{@twit}
% \skype{skype-id}
% \reddit{reddit-id}
% \medium{madium-id}
% \kaggle{kaggle-id}
% \hackerrank{hackerrank-id}
% \googlescholar{googlescholar-id}{name-to-display}
%% \firstname and \lastname will be used
% \googlescholar{googlescholar-id}{}
% \extrainfo{extra information}

\quote{``解码非解释, 抽象即理解"}

\bibliography{local_cn.bib,patents_misc_cn.bib}
%-------------------------------------------------------------------------------
\begin{document}

% Print the header with above personal information
% Give optional argument to change alignment(C: center, L: left, R: right)
\makecvheader[C]

% Print the footer with 3 arguments(<left>, <center>, <right>)
% Leave any of these blank if they are not needed
\makecvfooter
  {\today}
  {忻斌健~~~·~~~简历}
  {\thepage}


%-------------------------------------------------------------------------------
%	CV/RESUME CONTENT
%	Each section is imported separately, open each file in turn to modify content
%-------------------------------------------------------------------------------
\subfile{resume/summary_cn.tex}
\subfile{resume/experience_cn.tex}
% \documentclass[../cv.tex]{subfiles}

\begin{document}
%-------------------------------------------------------------------------------
%	SECTION TITLE
%-------------------------------------------------------------------------------
\cvsection{Honors \& Awards}


%-------------------------------------------------------------------------------
%	SUBSECTION TITLE
%-------------------------------------------------------------------------------
\cvsubsection{International Awards}


%-------------------------------------------------------------------------------
%	CONTENT
%-------------------------------------------------------------------------------
\begin{cvhonors}

%---------------------------------------------------------
  \cvhonor
    {2nd Place} % Award
    {AWS ASEAN AI/ML GameDay} % Event
    {Online} % Location
    {2021} % Date(s)

%---------------------------------------------------------
  \cvhonor
    {Finalist} % Award
    {DEFCON 28th CTF Hacking Competition World Final} % Event
    {Las Vegas, U.S.A} % Location
    {2020} % Date(s)

%---------------------------------------------------------
  \cvhonor
    {Finalist} % Award
    {DEFCON 26th CTF Hacking Competition World Final} % Event
    {Las Vegas, U.S.A} % Location
    {2018} % Date(s)

%---------------------------------------------------------
  \cvhonor
    {Finalist} % Award
    {DEFCON 25th CTF Hacking Competition World Final} % Event
    {Las Vegas, U.S.A} % Location
    {2017} % Date(s)

%---------------------------------------------------------
  \cvhonor
    {Finalist} % Award
    {DEFCON 22nd CTF Hacking Competition World Final} % Event
    {Las Vegas, U.S.A} % Location
    {2014} % Date(s)

%---------------------------------------------------------
  \cvhonor
    {Finalist} % Award
    {DEFCON 21st CTF Hacking Competition World Final} % Event
    {Las Vegas, U.S.A} % Location
    {2013} % Date(s)

%---------------------------------------------------------
  \cvhonor
    {Finalist} % Award
    {DEFCON 19th CTF Hacking Competition World Final} % Event
    {Las Vegas, U.S.A} % Location
    {2011} % Date(s)

%---------------------------------------------------------
  \cvhonor
    {6th Place} % Award
    {SECUINSIDE Hacking Competition World Final} % Event
    {Seoul, S.Korea} % Location
    {2012} % Date(s)

%---------------------------------------------------------
\end{cvhonors}


%-------------------------------------------------------------------------------
%	SUBSECTION TITLE
%-------------------------------------------------------------------------------
\cvsubsection{Domestic Awards}


%-------------------------------------------------------------------------------
%	CONTENT
%-------------------------------------------------------------------------------
\begin{cvhonors}

%---------------------------------------------------------
  \cvhonor
    {2nd Place} % Award
    {AWS Korea GameDay} % Event
    {Seoul, S.Korea} % Location
    {2021} % Date(s)

%---------------------------------------------------------
  \cvhonor
    {3rd Place} % Award
    {WITHCON Hacking Competition Final} % Event
    {Seoul, S.Korea} % Location
    {2015} % Date(s)

%---------------------------------------------------------
  \cvhonor
    {Silver Prize} % Award
    {KISA HDCON Hacking Competition Final} % Event
    {Seoul, S.Korea} % Location
    {2017} % Date(s)

%---------------------------------------------------------
  \cvhonor
    {Silver Prize} % Award
    {KISA HDCON Hacking Competition Final} % Event
    {Seoul, S.Korea} % Location
    {2013} % Date(s)

%---------------------------------------------------------
  \cvhonor
    {2nd Award} % Award
    {HUST Hacking Festival} % Event
    {S.Korea} % Location
    {2013} % Date(s)

%---------------------------------------------------------
  \cvhonor
    {3rd Award} % Award
    {HUST Hacking Festival} % Event
    {S.Korea} % Location
    {2010} % Date(s)

%---------------------------------------------------------
  \cvhonor
    {3rd Award} % Award
    {Holyshield 3rd Hacking Festival} % Event
    {S.Korea} % Location
    {2012} % Date(s)

%---------------------------------------------------------
  \cvhonor
    {2nd Award} % Award
    {Holyshield 3rd Hacking Festival} % Event
    {S.Korea} % Location
    {2011} % Date(s)

%---------------------------------------------------------
  \cvhonor
    {5th Place} % Award
    {PADOCON Hacking Competition Final} % Event
    {Seoul, S.Korea} % Location
    {2011} % Date(s)

%---------------------------------------------------------
\end{cvhonors}

%-------------------------------------------------------------------------------
%	SUBSECTION TITLE
%-------------------------------------------------------------------------------
\cvsubsection{Community}


%-------------------------------------------------------------------------------
%	CONTENT
%-------------------------------------------------------------------------------
\begin{cvhonors}

%---------------------------------------------------------
  \cvhonor
    {AWS Community Builder (Container)} % Award
    {Amazon Web Services (AWS)} % Event
    {} % Location
    {2022} % Date(s)

%---------------------------------------------------------
  \cvhonor
    {HashiCorp Ambassador} % Award
    {HashiCorp} % Event
    {} % Location
    {2022} % Date(s)

%---------------------------------------------------------
\end{cvhonors}
\end{document}

% \documentclass[../cv.tex]{subfiles}
\begin{document}
%-------------------------------------------------------------------------------
%	SECTION TITLE
%-------------------------------------------------------------------------------
\cvsection{Certificates}


%-------------------------------------------------------------------------------
%	CONTENT
%-------------------------------------------------------------------------------
\begin{cvhonors}

%---------------------------------------------------------
  \cvhonor
    {AWS Certified Advanced Networking - Specialty} % Name
    {Amazon Web Services (AWS)} % Issuer
    {} % Credential ID
    {2023} % Date(s)

%---------------------------------------------------------
  \cvhonor
    {AWS Certified Developer – Associate} % Name
    {Amazon Web Services (AWS)} % Issuer
    {} % Credential ID
    {2023} % Date(s)

%---------------------------------------------------------
  \cvhonor
    {AWS Certified Cloud Practitioner} % Name
    {Amazon Web Services (AWS)} % Issuer
    {} % Credential ID
    {2023} % Date(s)

%---------------------------------------------------------
  \cvhonor
    {AWS Certified Security - Specialty} % Name
    {Amazon Web Services (AWS)} % Issuer
    {} % Credential ID
    {2022} % Date(s)

%---------------------------------------------------------
  \cvhonor
    {AWS Certified Solutions Architect – Professional} % Name
    {Amazon Web Services (AWS)} % Issuer
    {} % Credential ID
    {2022} % Date(s)

%---------------------------------------------------------
  \cvhonor
    {AWS Certified Solutions Architect – Associate} % Name
    {Amazon Web Services (AWS)} % Issuer
    {} % Credential ID
    {2019} % Date(s)

%---------------------------------------------------------
  \cvhonor
    {AWS Certified SysOps Administrator – Associate} % Name
    {Amazon Web Services (AWS)} % Issuer
    {} % Credential ID
    {2021} % Date(s)

%---------------------------------------------------------
  \cvhonor
    {Certified Kubernetes Application Developer (CKAD)} % Name
    {The Linux Foundation} % Issuer
    {} % Credential ID
    {2020} % Date(s)

%---------------------------------------------------------
  \cvhonor
    {HashiCorp Certified: Consul Associate (002)} % Name
    {HashiCorp} % Issuer
    {} % Credential ID
    {2023} % Date(s)

%---------------------------------------------------------
  \cvhonor
    {HashiCorp Certified: Terraform Associate (003)} % Name
    {HashiCorp} % Issuer
    {} % Credential ID
    {2023} % Date(s)

%---------------------------------------------------------
  \cvhonor
    {HashiCorp Certified: Terraform Associate (002)} % Name
    {HashiCorp} % Issuer
    {} % Credential ID
    {2020} % Date(s)

%---------------------------------------------------------
\end{cvhonors}
\end{document}

% \documentclass[../cv.tex]{subfiles}

\begin{document}
%-------------------------------------------------------------------------------
%	SECTION TITLE
%-------------------------------------------------------------------------------
\cvsection{Presentation}


%-------------------------------------------------------------------------------
%	CONTENT
%-------------------------------------------------------------------------------
\begin{cventries}

%---------------------------------------------------------
  \cventry
    {Presenter for <Hosting Web Application for Free utilizing GitHub, Netlify and CloudFlare>} % Role
    {DevFest Seoul by Google Developer Group Korea} % Event
    {Seoul, S.Korea} % Location
    {Nov. 2017} % Date(s)
    {
      \begin{cvitems} % Description(s)
        \item {Introduced the history of web technology and the JAM stack which is for the modern web application development.}
        \item {Introduced how to freely host the web application with high performance utilizing global CDN services.}
      \end{cvitems}
    }

%---------------------------------------------------------
  \cventry
    {Presenter for <DEFCON 20th : The way to go to Las Vegas>} % Role
    {6th CodeEngn (Reverse Engineering Conference)} % Event
    {Seoul, S.Korea} % Location
    {Jul. 2012} % Date(s)
    {
      \begin{cvitems} % Description(s)
        \item {Introduced CTF(Capture the Flag) hacking competition and advanced techniques and strategy for CTF}
      \end{cvitems}
    }

%---------------------------------------------------------
  \cventry
    {Presenter for <Metasploit 101>} % Role
    {6th Hacking Camp - S.Korea} % Event
    {S.Korea} % Location
    {Sep. 2012} % Date(s)
    {
      \begin{cvitems} % Description(s)
        \item {Introduced basic procedure for penetration testing and how to use Metasploit}
      \end{cvitems}
    }

%---------------------------------------------------------
\end{cventries}
\end{document}

% \documentclass[../cv.tex]{subfiles}
\begin{document}
%-------------------------------------------------------------------------------
%	SECTION TITLE
%-------------------------------------------------------------------------------
\cvsection{Writing}


%-------------------------------------------------------------------------------
%	CONTENT
%-------------------------------------------------------------------------------
\begin{cventries}

%---------------------------------------------------------
  \cventry
    {Founder \& Writer} % Role
    {A Guide for Developers in Start-up} % Title
    {Facebook Page} % Location
    {Jan. 2015 - PRESENT} % Date(s)
    {
      \begin{cvitems} % Description(s)
        \item {Drafted daily news for developers in Korea about IT technologies, issues about start-up.}
      \end{cvitems}
    }

%---------------------------------------------------------
  \cventry
    {Undergraduate Student Reporter} % Role
    {AhnLab} % Title
    {S.Korea} % Location
    {Oct. 2012 - Jul. 2013} % Date(s)
    {
      \begin{cvitems} % Description(s)
        \item {Drafted reports about IT trends and Security issues on AhnLab Company magazine.}
      \end{cvitems}
    }

%---------------------------------------------------------
\end{cventries}
\end{document}

% \documentclass[../cv.tex]{subfiles}
\begin{document}
%-------------------------------------------------------------------------------
%	SECTION TITLE
%-------------------------------------------------------------------------------
\cvsection{Program Committees}


%-------------------------------------------------------------------------------
%	CONTENT
%-------------------------------------------------------------------------------
\begin{cvhonors}

%---------------------------------------------------------
  \cvhonor
    {Problem Writer} % Position
    {2016 CODEGATE Hacking Competition World Final} % Committee
    {S.Korea} % Location
    {2016} % Date(s)

%---------------------------------------------------------
  \cvhonor
    {Organizer \& Co-director} % Position
    {1st POSTECH Hackathon} % Committee
    {S.Korea} % Location
    {2013} % Date(s)

%---------------------------------------------------------
  \cvhonor
    {Staff} % Position
    {7th Hacking Camp} % Committee
    {S.Korea} % Location
    {2012} % Date(s)

%---------------------------------------------------------
  \cvhonor
    {Problem Writer} % Position
    {1st Hoseo University Teenager Hacking Competition} % Committee
    {S.Korea} % Location
    {2012} % Date(s)

%---------------------------------------------------------
  \cvhonor
    {Staff \& Problem Writer} % Position
    {JFF(Just for Fun) Hacking Competition} % Committee
    {S.Korea} % Location
    {2012} % Date(s)

%---------------------------------------------------------
\end{cvhonors}
\end{document}

\subfile{resume/education_cn.tex}
% \documentclass[../cv.tex]{subfiles}

\begin{document}
%-------------------------------------------------------------------------------
%	SECTION TITLE
%-------------------------------------------------------------------------------
\cvsection{Extracurricular Activity}


%-------------------------------------------------------------------------------
%	CONTENT
%-------------------------------------------------------------------------------
\begin{cventries}

%---------------------------------------------------------
  \cventry
    {Core Member} % Affiliation/role
    {B10S (B1t 0n the Security, Underground hacker team)} % Organization/group
    {S.Korea} % Location
    {Nov. 2011 - PRESENT} % Date(s)
    {
      \begin{cvitems} % Description(s) of experience/contributions/knowledge
        \item {Gained expertise in penetration testing areas, especially targeted on web application and software.}
        \item {Participated on a lot of hacking competition and won a good award.}
        \item {Held several hacking competitions non-profit, just for fun.}
      \end{cvitems}
    }

%---------------------------------------------------------
  \cventry
    {Member} % Affiliation/role
    {WiseGuys (Hacking \& Security research group)} % Organization/group
    {S.Korea} % Location
    {Jun. 2012 - PRESENT} % Date(s)
    {
      \begin{cvitems} % Description(s) of experience/contributions/knowledge
        \item {Gained expertise in hardware hacking areas from penetration testing on several devices including wireless router, smartphone, CCTV and set-top box.}
        \item {Trained wannabe hacker about hacking technique from basic to advanced and ethics for white hackers by hosting annual Hacking Camp.}
      \end{cvitems}
    }

%---------------------------------------------------------
  \cventry
    {Core Member \& President at 2013} % Affiliation/role
    {PoApper (Developers' Network of POSTECH)} % Organization/group
    {Pohang, S.Korea} % Location
    {Jun. 2010 - Jun. 2017} % Date(s)
    {
      \begin{cvitems} % Description(s) of experience/contributions/knowledge
        \item {Reformed the society focusing on software engineering and building network on and off campus.}
        \item {Proposed various marketing and network activities to raise awareness.}
      \end{cvitems}
    }

%---------------------------------------------------------
  \cventry
    {Member} % Affiliation/role
    {PLUS (Laboratory for UNIX Security in POSTECH)} % Organization/group
    {Pohang, S.Korea} % Location
    {Sep. 2010 - Oct. 2011} % Date(s)
    {
      \begin{cvitems} % Description(s) of experience/contributions/knowledge
        \item {Gained expertise in hacking \& security areas, especially about internal of operating system based on UNIX and several exploit techniques.}
        \item {Participated on several hacking competition and won a good award.}
        \item {Conducted periodic security checks on overall IT system as a member of POSTECH CERT.}
        \item {Conducted penetration testing commissioned by national agency and corporation.}
      \end{cvitems}
    }

%---------------------------------------------------------
  \cventry
    {Member} % Affiliation/role
    {MSSA (Management Strategy Club of POSTECH)} % Organization/group
    {Pohang, S.Korea} % Location
    {Sep. 2013 - Jun. 2017} % Date(s)
    {
      \begin{cvitems} % Description(s) of experience/contributions/knowledge
        \item {Gained knowledge about several business field like Management, Strategy, Financial and marketing from group study.}
        \item {Gained expertise in business strategy areas and inisght for various industry from weekly industry analysis session.}
      \end{cvitems}
    }

%---------------------------------------------------------
\end{cventries}
\end{document}

\subfile{resume/skills_cn.tex}

%-------------------------------------------------------------------------------
% SECTION TITLE
%-------------------------------------------------------------------------------
\cvsection{出版物}

%\printbibliography

%-------------------------------------------------------------------------------
% SUBSECTION TITLE
%-------------------------------------------------------------------------------
%    \nocite{gillies}
%    \nocite{glashow}
%    \nocite{herrman}
%
    %\ifSubfilesClassLoaded{%
    %    \printbibliography[
    %        heading=none,
    %        sorting=ydnt
    %    ]
    %  }{} % we have no 'else' action

\nocite{xin2009multiscale}
\nocite{xin2007evaluation}
\nocite{xin2002KneeSimulator}


\nocite{xin2008diss}

\nocite{Xin_VEOS_22}
\nocite{Xin_GenAI_23}
\nocite{Xin_VLM_24}
\nocite{Xin_Latent_Diffusion_23}
\nocite{Xin_Daimler_08}

\printbibliography[
heading=none,
sorting=ydnt
]

%-------------------------------------------------------------------------------
\end{document}
