%!TEX TS-program = xelatex
%!TEX encoding = UTF-8 Unicode
% Awesome CV LaTeX Template for CV/Resume
%
% This template has been downloaded from:
% https://github.com/posquit0/Awesome-CV
%
% Author:
% Claud D. Park <posquit0.bj@gmail.com>
% http://www.posquit0.com
%
% Template license:
% CC BY-SA 4.0 (https://creativecommons.org/licenses/by-sa/4.0/)
%


%!TEX TS-program = xelatex
%!TEX encoding = UTF-8 Unicode
% 很棒的简历 LaTeX 模板
%
% 该模板来自:
% https://github.com/posquit0/Awesome-CV
%
% 作者:
% Claud D. Park <posquit0.bj@gmail.com>
% http://www.posquit0.com
%
% 模板许可证:
% CC BY-SA 4.0 (https://creativecommons.org/licenses/by-sa/4.0/)
%

%-------------------------------------------------------------------------------
% 配置
%-------------------------------------------------------------------------------
% 默认情况下为 A4 纸张大小,使用 'letterpaper' 选项可切换为 US letter
\documentclass[11pt, a4paper]{awesome-cv-cn}

\usepackage{xeCJK}
\newfontfamily\hei{Adobe Heiti Std}
\newfontfamily\kai{Adobe Kaiti Std}
\setCJKmainfont[BoldFont=LXGWWenKai-Bold,ItalicFont=LXGWWenKai-Light]{LXGW WenKai}   % 设置默认中文字体
\setCJKmonofont{LXGWWenKaiMono}   % 设置等宽字体
\fontdir[fonts/]

\usepackage[style=numeric,defernumbers=true,maxbibnames=99,sorting=ydnt]{biblatex}
\usepackage{subfiles}
\usepackage{twemojis}
% 使用 geometry 配置页面边距
\geometry{left=1.4cm, top=.8cm, right=1.4cm, bottom=1.8cm, footskip=.5cm}

% 高亮颜色
% 可选颜色: awesome-emerald, awesome-skyblue, awesome-red, awesome-pink, awesome-orange
%           awesome-nephritis, awesome-concrete, awesome-darknight
\colorlet{awesome}{awesome-red}
% 如果想指定自己的颜色,取消注释下面一行
% \definecolor{awesome}{HTML}{CA63A8}

% 文本颜色
% 如果想指定自己的颜色,取消注释下面几行
% \definecolor{darktext}{HTML}{414141}
% \definecolor{text}{HTML}{333333}
% \definecolor{graytext}{HTML}{5D5D5D}
% \definecolor{lighttext}{HTML}{999999}
% \definecolor{sectiondivider}{HTML}{5D5D5D}

% 设置为 false 如果不想用 awesome 颜色高亮段落
\setbool{acvSectionColorHighlight}{true}

% 如果想更改社交信息分隔符,从 '|' 更改为其他内容
\renewcommand{\acvHeaderSocialSep}{\quad\textbar\quad}


%-------------------------------------------------------------------------------
% 个人信息
% 注释掉下面的任何行如果不需要
%-------------------------------------------------------------------------------
% 可用选项: circle|rectangle,edge/noedge,left/right
\photo{avatar1.jpg}
\name{忻}{斌健}
\position{深度学习{\enskip\cdotp\enskip}自动驾驶{\enskip\cdotp\enskip}软件设计}
\address{上海市青浦区巷居路99弄288号102室}

\mobile{(+86) 139-1896-1550}
\email{binjian.xin@hotmail.com}
%\dateofbirth{January 1st, 1970}
\homepage{binjian.github.io/zh}
\github{binjian}
\linkedin{binjian-xin}
%\twitter{xinbinjian}
%% \firstname 和 \lastname 将会被使用
% \googlescholar{googlescholar-id}{}
% \extrainfo{extra information}

\quote{``解码非解释, 抽象即理解"}
%-------------------------------------------------------------------------------
% 参考文献
%-------------------------------------------------------------------------------
%\addbibresource{biblatex-examples.bib}
%\addbibresource{cv/references.bib}
%\addbibresource{patents.bib}
\addbibresource{local_cn.bib}
\addbibresource{patents_misc_cn.bib}
%-------------------------------------------------------------------------------
\begin{document}

% 打印包含个人信息的页眉
% 可选参数可以改变对齐方式(C: center, L: left, R: right)
\makecvheader

% 打印包含日期、姓名的页脚
% 留空如果不需要
\makecvfooter
{\today}
{忻斌健~~~·~~~简历}
{\thepage}


%-------------------------------------------------------------------------------
% 简历内容
% 每个部分分别导入,依次打开每个文件以修改内容
%-------------------------------------------------------------------------------
\subfile{cv/education_cn.tex}
\subfile{cv/experience_cn.tex}
\subfile{cv/skills_cn.tex}

%\subfile{cv/extracurricular.tex}
%\subfile{cv/honors.tex}
%\subfile{cv/certificates.tex}
%\subfile{cv/presentation.tex}
%\subfile{cv/writing.tex}
% 对于来自 bibtex 的出版物,添加到 cv/references.bib 并取消下面的注释
%\subfile{cv/publications.tex}

%-------------------------------------------------------------------------------
% 章节标题
%-------------------------------------------------------------------------------
\cvsection{出版物}

%\printbibliography

%-------------------------------------------------------------------------------
% 子章节标题
%-------------------------------------------------------------------------------
\cvsubsection{期刊}

\begin{refsegment}
	%    \nocite{gillies}
	%    \nocite{glashow}
	%    \nocite{herrman}
	%
	%\ifSubfilesClassLoaded{%
	%    \printbibliography[
	%        heading=none,
	%        sorting=ydnt
	%    ]
	%  }{} % we have no 'else' action

	\nocite{xin2009multiscale}
	\nocite{xin2004bildfolgenauswertung}
	\nocite{xin2002AntColony}

	\printbibliography[
		segment=\therefsegment,
		heading=none,
		sorting=ydnt
	]
\end{refsegment}

%-------------------------------------------------------------------------------
% 子章节标题
%-------------------------------------------------------------------------------
\cvsubsection{会议}

\begin{refsegment}

	\nocite{xin2007evaluation}
	\nocite{xin2002KneeSimulator}
	\printbibliography[
		segment=\therefsegment,
		heading=none,
		sorting=ydnt
	]
\end{refsegment}


%-------------------------------------------------------------------------------
% 子章节标题
%-------------------------------------------------------------------------------
\cvsubsection{书}

\begin{refsegment}

	\nocite{xin2008diss}

	\printbibliography[
		segment=\therefsegment,
		heading=none,
		sorting=ydnt
	]
\end{refsegment}
%-------------------------------------------------------------------------------
% 子章节标题
%-------------------------------------------------------------------------------
\cvsubsection{专利}

\begin{refsegment}

	\nocite{Xin_VEOS_22}
	\nocite{Xin_GenAI_23}
	\nocite{Xin_LLM_24}
	\nocite{Xin_VLM_24}
	\nocite{Xin_Latent_Diffusion_23}
	\nocite{Xin_Fu_Pan_Simulation_Test_RL_22}
	\nocite{Xin_Chen_NN_TSFeatures_23}
	\nocite{Pan_Xin_DrvStyle_23}
	\nocite{Xin_RearView_17}
	\nocite{Xin_Fu_Pan_Simulation_Test_RL_22}
	\nocite{Xin_Daimler_08}

	\printbibliography[
		segment=\therefsegment,
		heading=none,
		sorting=ydnt
	]
\end{refsegment}

%
%\subfile{cv/committees_xin.tex}
%-------------------------------------------------------------------------------
%-------------------------------------------------------------------------------
\end{document}
