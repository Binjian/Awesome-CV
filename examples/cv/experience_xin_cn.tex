\documentclass[../cv_xin_cn.tex]{subfiles}

%\bibliography{../local.bib,../patents_misc.bib}
%\addbibresource{../local.bib}
%\addbibresource{../patents_misc.bib}
\begin{document}
%-------------------------------------------------------------------------------
%	SECTION TITLE
%-------------------------------------------------------------------------------
\cvsection{工作经验}


%-------------------------------------------------------------------------------
%	CONTENT
%-------------------------------------------------------------------------------
\begin{cventries}

%---------------------------------------------------------
  \cventry
    {高级技术总监} % Job title
    {前晨汽车} % Organization
    {中国上海} % Location
    {2020年11月 - 2024年5月} % Date(s)
    {
      \begin{cvitems} % Description(s) of tasks/responsibilities
          \item 基于强化学习的多模式复杂环境中的电动力总成控制优化
          \hyperlink{Xin_VEOS_22}{$\rightarrow$\circled{P}7} \hyperlink{Xin_Fu_Pan_Simulation_Test_RL_22}{\circled{P}8} \hyperlink{Pan_Xin_DrvStyle_23}{\circled{P}3}.
          \item 基于生成模型的时间序列异常检测和电池安全状态预测。 \hyperlink{Xin_GenAI_23}{$\rightarrow$\circled{P}5} \hyperlink{Xin_Chen_NN_TSFeatures_23}{\circled{P}6}
          \item 在自动驾驶中应用多模态基础模型和大型语言模型 \hyperlink{Xin_LLM_24}{\circled{P}1} \hyperlink{Xin_VLM_24}{\circled{P}2} \hyperlink{Xin_Latent_Diffusion_23}{\circled{P}4}.
          \item 流式数据管道的软件设计与开发,用于时间序列:
            \begin{itemize}
                    \item 在线深度强化学习数据流 (ETL, 深度学习训练/推理管道) \href{https://binjian.github.io/tspace/}{$\rightarrow$ \textbf{tspace}},
                    \item CAN 应用程序包 \href{https://binjian.github.io/candycan/}{$\rightarrow$ \textbf{candycan}},
                    \item 带有生成式人工智能的时间序列分析 \href{https://github.io/binjian/funes-ts/}{$\rightarrow$ \textbf{funes-ts}}.
                    %\item \href{https://binjian.gitlab.io/tspace/}{tspace}
            \end{itemize}
      \end{cvitems}
    }

%---------------------------------------------------------
  \cventry
    {高级经理} % Job title
    {蔚来} % Organization
    {中国上海} % Location
    {2017年11月 - 2020年11月} % Date(s)
    {
      \begin{cvitems}
        \item L4级自动驾驶系统的先进硬件和软件设计。
        \item 团队(10+工程师)组建和发展管理。
        \item 车队:10+辆配备L4级传感器配置和计算平台的电动车辆。
        \item 智能充电和自动停车辅助系统(公共资助项目)。
        \item 智能互联汽车(ICV)道路测试在上海和北京的许可申请和运营。
        \item 在北京进行T3许可测试里程的前三名。
        \item 2019年海南博鳌论坛的5G ICV演示。
      \end{cvitems}
    }

%---------------------------------------------------------
  \cventry
    {技术经理} % Job title
    {PATAC/上汽通用} % Organization
    {中国上海} % Location
    {2015年10月 - 2017年11月} % Date(s)
    {
      \begin{cvitems}
        \item 主动安全领域单元(ADU)的系统与软件架构设计。
        \item PATAC ADU A 样品:嵌入式平台的系统和软件架构。
        \item SAIC-MAXUS SV73高速辅助的软件架构。
        \item 基于摄像头的驾驶员监控系统。
        \item 全景摄像头系统 \hyperlink{Xin_RearView_17}{\circled{P}9}.
      \end{cvitems}
    }


%---------------------------------------------------------
  \cventry
    {软件经理} % Job title
    {维世特亚太} % Organization
    {中国上海} % Location
    {2015年1月 - 2015年8月} % Date(s)
    {
      \begin{cvitems}
        \item 仪表板的SOP项目。
      \end{cvitems}
    }

%---------------------------------------------------------
  \cventry
    {高级经理} % Job title
    {海拉上海电子} % Organization
    {中国上海} % Location
    {2014年7月 - 2015年1月} % Date(s)
    {
      \begin{cvitems}
        \item BCM和PEPS的SOP项目。
        \item PEPS、BCM、BSW的平台项目。
      \end{cvitems}
    }

%---------------------------------------------------------
  \cventry
    {高级经理} % Job title
    {哈曼上海研发中心} % Organization
    {中国上海} % Location
    {2009年9月 - 2014年7月} % Date(s)
    {
      \begin{cvitems}
        \item 视频驱动的ADAS系统开发。
        \item 基于摄像头的停车系统的SOP项目:
          \begin{itemize}
            \item 吉利KC-1的3D全景系统(SVS)的SOP,
            \item 吉利、双龙、塔塔、通用、铃木、现代和大众的后视摄像头部署(SOP),
            \item 与超声波停车辅助的融合。
           \end{itemize}
        \item ADAS先进研究的监督:
          \begin{itemize}
            \item 在信息娱乐平台上的LDW和FCW,
            \item 增强型导航,
            \item 运动物体检测。
            \item 环绕视图演示系统(机器人车和OEM车辆)和演示(CES、日内瓦车展)的设计。
          \end{itemize}
      \end{cvitems}
    }
%---------------------------------------------------------
\end{cventries}
\end{document}
