\documentclass[../cv_cn.tex]{subfiles}

%\bibliography{../local.bib,../patents_misc.bib}
%\addbibresource{../local_cn.bib}
%\addbibresource{../patents_misc_cn.bib}
\begin{document}
%-------------------------------------------------------------------------------
%	SECTION TITLE
%-------------------------------------------------------------------------------
\cvsection{工作经验}


%-------------------------------------------------------------------------------
%	CONTENT
%-------------------------------------------------------------------------------
\begin{cventries}

	%---------------------------------------------------------
	\cventry
	{高级技术总监,智能系统} % Job title
	{前晨汽车} % Organization
	{中国上海} % Location
	{2020年11月 - 2024年5月} % Date(s)
	{
		\begin{cvitems} % Description(s) of tasks/responsibilities
			\item 基于深度强化学习的运动规划(能效提升20\%) \supercite{Xin_VEOS_22} \supercite{Xin_Fu_Pan_Simulation_Test_RL_22} \supercite{Pan_Xin_DrvStyle_23},
			\begin{itemize}
				\item 基于DDPG, TD3, SAC的在线和离线两种模式下的车辆规划算法应用与分析
				\item 复杂POMDP模型下的序列深度模型RDPG(LSTM)应用与分析
				\item Multi-Agent分布式训练与联邦学习
				\item 基于深度扩散模型Q-Learning(IDQL,QSM)的离线行为模式识别与应用
				\item 独立开发深度强化学习应用软件 (ETL与ML协同异步管道架构设计实现) $\rightarrow$ \textbf{tspace} \href{https://binjian.github.io/tspace/}{\faGithub},
				\item 集成Tensorflow与Flax/Jax框架
				\item CAN应用程序包(集成Pydantic, Github Actions虚拟测试,文件动态加载,动态缓存) $\rightarrow$ \textbf{candycan} \href{https://binjian.github.io/candycan/}{\faGithub},
			\end{itemize}
			\item 基于生成模型的时间序列预测与应用 \supercite{Xin_GenAI_23} \supercite{Xin_Chen_NN_TSFeatures_23},
			\begin{itemize}
				\item 基于对抗生成网络(GAN)的多模态时间序列无监督学习
				\item 集成Pytorch与LSTM,Transformer架构
				\item 独立开发基于生成模型的时间序列分析应用软件 $\rightarrow$ \textbf{funes-ts} \href{https://github.com/binjian/funes-ts/}{\faGithub}。
				\item 模型蒸馏与数据蒸馏
			\end{itemize}
			\item 自动驾驶多模态基础模型和大型语言模型的研究与应用
			\begin{itemize}
				\item 使用自然语言接口基于LLM与多模态基础模型的自动驾驶决策系统 \supercite{Xin_LLM_24} \supercite{Xin_VLM_24}
				\item 基于隐空间扩散模型的前视摄像头图像复原\supercite{Xin_Latent_Diffusion_23}
				\item NERF及Gaussian Splatting驾驶场景三维建模
			\end{itemize}
		\end{cvitems}
	}

	%---------------------------------------------------------
	\cventry
	{高级经理,自动驾驶} % Job title
	{蔚来汽车} % Organization
	{中国上海} % Location
	{2017年11月 - 2020年11月} % Date(s)
	{
		\begin{cvitems}
			\item L4级自动驾驶系统硬件和软件设计开发
			\begin{itemize}
				\item 基于动态规划,混合A*算法/RRT的路由算法
				\item 满足Reeds-Shepp约束的泊车路由规划
				\item 模型预测控制的轨迹规划
				\item 深度神经网络的视频和点云目标检测与数据融合
				\item 摄像头标定以及与激光雷达联合标定
				\item 基于ROS自动驾驶系统集成与开发
			\end{itemize}
			\item 智能充电和自动泊车系统(上海市科委项目),
			\item 智能网联汽车(ICV)道路测试,上海和北京的测试许可申请和运营,
		\end{cvitems}
	}

	%---------------------------------------------------------
	\cventry
	{技术经理,ADAS} % Job title
	{泛亚/上汽通用} % Organization
	{中国上海} % Location
	{2015年10月 - 2017年11月} % Date(s)
	{
		\begin{cvitems}
			\item 主动安全域控制器(ADU)的系统与软件架构设计,
			\item ADU A 样:嵌入式平台的系统和软件架构,
			\item SAIC-MAXUS SV73高速辅助的软件架构,
			\item 基于摄像头的驾驶员监控系统,
			\item 全景影像系统 \supercite{Xin_RearView_17}。
		\end{cvitems}
	}


	%---------------------------------------------------------
	\cventry
	{软件经理} % Job title
	{伟世通亚太} % Organization
	{中国上海} % Location
	{2015年1月 - 2015年8月} % Date(s)
	{
		\begin{cvitems}
			\item 仪表板的量产项目。
		\end{cvitems}
	}

	%---------------------------------------------------------
	\cventry
	{高级软件经理} % Job title
	{海拉电子} % Organization
	{中国上海} % Location
	{2014年7月 - 2015年1月} % Date(s)
	{
		\begin{cvitems}
			\item BCM和PEPS的量产项目。
			\item PEPS、BCM、BSW的平台项目。
		\end{cvitems}
	}

	%---------------------------------------------------------
	\cventry
	{高级经理,ADAS} % Job title
	{哈曼研发中心} % Organization
	{中国上海} % Location
	{2009年9月 - 2014年7月} % Date(s)
	{
		\begin{cvitems}
			\item 视频ADAS系统开发,
			\item 基于摄像头的泊车系统的量产项目:
			\item ADAS前期研究开发:
			\begin{itemize}
				\item 信息娱乐平台上的LDW和FCW,
				\item 增强型导航,
				\item 运动物体检测,
				\item 环视演示系统的设计(机器人车和OEM车辆)和展会(CES,日内瓦车展)。
			\end{itemize}
		\end{cvitems}
	}
	%---------------------------------------------------------
\end{cventries}
\end{document}
