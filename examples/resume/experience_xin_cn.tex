\documentclass[../resume_xin.tex]{subfiles}

%\bibliography{../local.bib,../patents_misc.bib}
%\addbibresource{../local.bib}
%\addbibresource{../patents_misc.bib}
\begin{document}
%-------------------------------------------------------------------------------
%	SECTION TITLE
%-------------------------------------------------------------------------------
\cvsection{工作经历}


%-------------------------------------------------------------------------------
%	CONTENT
%-------------------------------------------------------------------------------
\begin{cventries}

%---------------------------------------------------------
  \cventry
    {高级技术总监} % Job title
    {前晨汽车} % Organization
    {中国上海} % Location
    {2020年11月 - 2024年5月} % Date(s)
    {
      \begin{cvitems} % Description(s) of tasks/responsibilities
          \item 基于强化学习的多模式复杂环境电动力总成控制优化(能效提升10\%) \supercite{Xin_VEOS_22},
          \item 基于生成模型的时间序列异常检测和电池安全状态预测 \supercite{Xin_GenAI_23},
          \item 自动驾驶多模态基础模型和大型语言模型的研究与应用 \supercite{Xin_VLM_24} \supercite{Xin_Latent_Diffusion_23},
          \item 时间序列深度学习数据软件设计与开发:
            \begin{itemize}
                    \item 在线深度强化学习数据流 (ETL, 深度学习训练/推理数据管道) $\rightarrow$ \textbf{tspace} \href{https://binjian.github.io/tspace/}{\twemoji{link}},
                    \item CAN 应用程序包 $\rightarrow$ \textbf{candycan} \href{https://binjian.github.io/candycan/}{\twemoji{link}},
                    \item 基于生成模型的时间序列分析 $\rightarrow$ \textbf{funes-ts} \href{https://github.io/binjian/funes-ts/}{\twemoji{link}}。
                    %\item \href{https://binjian.gitlab.io/tspace/}{tspace}
            \end{itemize}
      \end{cvitems}
    }

%---------------------------------------------------------
  \cventry
    {高级经理} % Job title
    {蔚来汽车} % Organization
    {中国上海} % Location
    {2017年11月 - 2020年11月} % Date(s)
    {
      \begin{cvitems}
        \item L4级自动驾驶系统硬件和软件设计,
        \item 团队(10+工程师)组建和发展管理,
        \item 车队:10+辆配备L4级传感器和计算平台的电动车,
        \item 智能充电和自动泊车系统(上海市科委项目),
        \item 智能网联汽车(ICV)道路测试,上海和北京的测试许可申请和运营,
        \item 北京T3许可证测试里程的前三名,
        \item 2019年海南博鳌论坛的5G ICV演示。
      \end{cvitems}
    }

%---------------------------------------------------------
  \cventry
    {技术经理} % Job title
    {泛亚/上汽通用} % Organization
    {中国上海} % Location
    {2015年10月 - 2017年11月} % Date(s)
    {
      \begin{cvitems}
        \item 主动安全域控制器(ADU)的系统与软件架构设计,
        \item ADU A 样:嵌入式平台的系统和软件架构,
        \item SAIC-MAXUS SV73高速辅助的软件架构,
        \item 基于摄像头的驾驶员监控系统,
        \item 全景影像系统 \supercite{Xin_RearView_17}。
      \end{cvitems}
    }


%---------------------------------------------------------
  \cventry
    {软件经理} % Job title
    {伟世通亚太} % Organization
    {中国上海} % Location
    {2015年1月 - 2015年8月} % Date(s)
    {
      \begin{cvitems}
        \item 仪表板的量产项目。
      \end{cvitems}
    }

%---------------------------------------------------------
  \cventry
    {高级经理} % Job title
    {海拉电子} % Organization
    {中国上海} % Location
    {2014年7月 - 2015年1月} % Date(s)
    {
      \begin{cvitems}
        \item BCM和PEPS的量产项目。
        \item PEPS、BCM、BSW的平台项目。
      \end{cvitems}
    }

%---------------------------------------------------------
  \cventry
    {高级经理} % Job title
    {哈曼研发中心} % Organization
    {中国上海} % Location
    {2009年9月 - 2014年7月} % Date(s)
    {
      \begin{cvitems}
        \item 视频ADAS系统开发,
        \item 基于摄像头的泊车系统的量产项目:
          \begin{itemize}
            \item 吉利KC-1的3D环视系统(SVS)量产,
            \item 吉利、双龙、塔塔、通用、铃木、现代和大众的倒车影像系统量产。
           \end{itemize}
        \item ADAS前期研究开发:
          \begin{itemize}
            \item 信息娱乐平台上的LDW和FCW,
            \item 增强型导航,
            \item 运动物体检测,
            \item 环视演示系统的设计(机器人车和OEM车辆)和展会(CES,日内瓦车展)。
          \end{itemize}
      \end{cvitems}
    }
%---------------------------------------------------------
\end{cventries}
\end{document}
