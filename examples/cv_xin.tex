%!TEX TS-program = xelatex
%!TEX encoding = UTF-8 Unicode
% Awesome CV LaTeX Template for CV/Resume
%
% This template has been downloaded from:
% https://github.com/posquit0/Awesome-CV
%
% Author:
% Claud D. Park <posquit0.bj@gmail.com>
% http://www.posquit0.com
%
% Template license:
% CC BY-SA 4.0 (https://creativecommons.org/licenses/by-sa/4.0/)
%


%-------------------------------------------------------------------------------
% CONFIGURATIONS
%-------------------------------------------------------------------------------
% A4 paper size by default, use 'letterpaper' for US letter
\documentclass[11pt, a4paper]{awesome-cv}
\usepackage[style=numeric,defernumbers=true,maxbibnames=99,sorting=ydnt]{biblatex}
\usepackage{subfiles}
\usepackage{twemojis}
%\usepackage{tikz}
% Configure page margins with geometry
\geometry{left=1.4cm, top=.8cm, right=1.4cm, bottom=1.8cm, footskip=.5cm}

%\newcommand*\circled[1]{\tikz[baseline=(char.base)]{
%            \node[shape=circle,draw,thick,inner sep=1pt] (char) {#1};}}

% Color for highlights
% Awesome Colors: awesome-emerald, awesome-skyblue, awesome-red, awesome-pink, awesome-orange
%                 awesome-nephritis, awesome-concrete, awesome-darknight
\colorlet{awesome}{awesome-red}
% Uncomment if you would like to specify your own color
% \definecolor{awesome}{HTML}{CA63A8}

% Colors for text
% Uncomment if you would like to specify your own color
% \definecolor{darktext}{HTML}{414141}
% \definecolor{text}{HTML}{333333}
% \definecolor{graytext}{HTML}{5D5D5D}
% \definecolor{lighttext}{HTML}{999999}
% \definecolor{sectiondivider}{HTML}{5D5D5D}

% Set false if you don't want to highlight section with awesome color
\setbool{acvSectionColorHighlight}{true}

% If you would like to change the social information separator from a pipe (|) to something else
\renewcommand{\acvHeaderSocialSep}{\quad\textbar\quad}


%-------------------------------------------------------------------------------
%	PERSONAL INFORMATION
%	Comment any of the lines below if they are not required
%-------------------------------------------------------------------------------
% Available options: circle|rectangle,edge/noedge,left/right
\photo{avatar1.jpg}
\name{Binjian}{Xin}
\position{Deep Learning{\enskip\cdotp\enskip}Autonomous Driving{\enskip\cdotp\enskip}Software Design}
\address{Rm. 102, Bd. 288, 99 Long Xiangju Road, Qingpu, Shanghai}

\mobile{(+86) 139-1896-1550}
\email{binjian.xin@hotmail.com}
%\dateofbirth{January 1st, 1970}
\homepage{binjian.github.io}
\github{binjian}
\linkedin{binjian-xin}
\twitter{xinbinjian}
%% \firstname and \lastname will be used
% \googlescholar{googlescholar-id}{}
% \extrainfo{extra information}

\quote{``$Decoding \neq Interpretation$, $Abstraction = Understanding$."}
%-------------------------------------------------------------------------------
%	BIBLIOGRAPHY
%-------------------------------------------------------------------------------
%\addbibresource{biblatex-examples.bib}
%\addbibresource{cv/references.bib}
%\addbibresource{patents.bib}
\bibliography{local.bib,patents_misc.bib}
%-------------------------------------------------------------------------------
\begin{document}

% Print the header with above personal information
% Give optional argument to change alignment(C: center, L: left, R: right)
\makecvheader

% Print the footer with 3 arguments(<left>, <center>, <right>)
% Leave any of these blank if they are not needed
\makecvfooter
  {\today}
  {Binjian Xin~~~·~~~Curriculum Vitae}
  {\thepage}


%-------------------------------------------------------------------------------
%	CV/RESUME CONTENT
%	Each section is imported separately, open each file in turn to modify content
%-------------------------------------------------------------------------------
\subfile{cv/education_xin.tex}
\subfile{cv/experience_xin.tex}
\subfile{cv/skills_xin.tex}
%\subfile{cv/extracurricular.tex}
%\subfile{cv/honors.tex}
%\subfile{cv/certificates.tex}
%\subfile{cv/presentation.tex}
%\subfile{cv/writing.tex}
% For publications from bibtex, add to cv/references.bib and uncomment below
%\subfile{cv/publications.tex}

%-------------------------------------------------------------------------------
% SECTION TITLE
%-------------------------------------------------------------------------------
\cvsection{Publications}

%\printbibliography

%-------------------------------------------------------------------------------
% SUBSECTION TITLE
%-------------------------------------------------------------------------------
\cvsubsection{Journal Articles}

\begin{refsegment}
%    \nocite{gillies}
%    \nocite{glashow}
%    \nocite{herrman}
%
    %\ifSubfilesClassLoaded{%
    %    \printbibliography[
    %        heading=none,
    %        sorting=ydnt
    %    ]
    %  }{} % we have no 'else' action

 \nocite{xin2009multiscale}
 \nocite{xin2004bildfolgenauswertung}
 \nocite{xin2002AntColony}

    \printbibliography[
    segment=\therefsegment,
    heading=none,
    sorting=ydnt
    ]
\end{refsegment}

%-------------------------------------------------------------------------------
% SUBSECTION TITLE
%-------------------------------------------------------------------------------
\cvsubsection{Conference Proceedings}

\begin{refsegment}

\nocite{xin2007evaluation}
\nocite{xin2002KneeSimulator}
    \printbibliography[
    segment=\therefsegment,
    heading=none,
    sorting=ydnt
    ]
\end{refsegment}

%-------------------------------------------------------------------------------
% SUBSECTION TITLE
%-------------------------------------------------------------------------------
\cvsubsection{Book}

\begin{refsegment}

\nocite{xin2008diss}

    \printbibliography[
    segment=\therefsegment,
    heading=none,
    sorting=ydnt
    ]
\end{refsegment}
%-------------------------------------------------------------------------------
% SUBSECTION TITLE
%-------------------------------------------------------------------------------
\cvsubsection{Patents}

\begin{refsegment}
\nocite{Xin_VEOS_22}
\nocite{Xin_GenAI_23}
\nocite{Xin_LLM_24}
\nocite{Xin_VLM_24}
\nocite{Xin_Latent_Diffusion_23}
\nocite{Xin_Fu_Pan_Simulation_Test_RL_22}
\nocite{Xin_Chen_NN_TSFeatures_23}
\nocite{Pan_Xin_DrvStyle_23}
\nocite{Xin_RearView_17}
\nocite{Xin_Fu_Pan_Simulation_Test_RL_22}
\nocite{Xin_Daimler_08}

    \printbibliography[
    segment=\therefsegment,
    heading=none,
    sorting=ydnt
    ]
\end{refsegment}

%
%\subfile{cv/committees_xin.tex}
%-------------------------------------------------------------------------------
%-------------------------------------------------------------------------------
\end{document}
